\documentclass{article}

\usepackage{fancyhdr}
\usepackage{extramarks}
\usepackage{amsmath}
\usepackage{amsthm}
\usepackage{amsfonts}
\usepackage{tikz}
\usepackage[plain]{algorithm}
\usepackage{algpseudocode}
\usepackage{hyperref}

\usetikzlibrary{automata,positioning}

%
% Basic Document Settings
%

\topmargin=-0.45in
\evensidemargin=0in
\oddsidemargin=0in
\textwidth=6.5in
\textheight=9.0in
\headsep=0.25in

\linespread{1.1}

\pagestyle{fancy}
\lhead{\hmwkAuthorName}
\chead{\hmwkClass\ : \hmwkTitle}
\rhead{\hmwkClassInstructor}
\cfoot{\thepage}

\renewcommand\headrulewidth{0.4pt}
\renewcommand\footrulewidth{0.4pt}

\setlength\parindent{0pt}
%
% Homework Details
%   - Title
%   - Due date
%   - Class
%   - Section/Time
%   - Instructor
%   - Author
%

\newcommand{\hmwkTitle}{Report Week 1}
\newcommand{\hmwkDueDate}{January 20, 2020}
\newcommand{\hmwkClass}{URMP}
\newcommand{\hmwkClassInstructor}{Panayot Vassilevski}
\newcommand{\hmwkAuthorName}{\textbf{Austen Nelson}}

%
% Title Page
%

\title{
    \vspace{2in}
    \textmd{\textbf{\hmwkClass:\ \hmwkTitle}}\\
    \normalsize\vspace{0.1in}\small{\hmwkDueDate}\\
    \vspace{0.1in}\large{\textit{\hmwkClassInstructor}}
    \vspace{3in}
}

\author{\hmwkAuthorName}
\date{}

\begin{document}

\maketitle
\pagebreak

\section{Introduction}
This is still the early stages of my research so most of my progress is reading related materials and coming up with a more constructed direction and plan. I read a few related papers and am in the progress of creating an annotated bibliography for self reference. In this process I have come across quality inspiration.

\section{Progress}
The first step in my actual research will be doing data collection of recipes. A few of the papers I read referenced a recipe database called "open recipes" but when I visited the site the database seems to be removed. I have emailed the administrators to see if I can get access to the database but haven't heard back yet. This isn't a huge problem as I can scrape this data myself. I have started with allrecipes.com. This week I wrote a basic scraping tool and have accumulated about 20,000 recipes. For this project I would like to implement clustering algorithms designed for very large graphs so I would like to have 200,000+ recipes. To accumulate this data will take some time but it should be possible.

\section{Plans}
One of the first graphs I plan on constructing is an ingredient --- recipe graph where each ingredient and recipe is a node and an edges exist between a recipe and an ingredient if the ingredient is present in the recipe. This edge will also be weighted with the quantity. This is a nice bipartite graph and should have some nice properties to investigate. Another graph I plan on constructing will be the ingredient co-occurrence graph with ingredients as nodes and edges between ingredients that show up in recipes together. The weight of those edges will be the number of recipes with the co-occurrence. When representing this graph as an adjacency matrix the result is a nice symmetric matrix that can be clustered using modularity functionals.

\section{Potential Issues}
Now that I have some starting data I can move on to the next portion of my project of starting to create these graphs. The first challenge will be processing ingredient lists of recipes into sensible abstractions. Originally I was considering using regular expressions to try and pull out the quantity, unit, ingredient, comments, and excess information from each ingredient but quickly realized this cannot be done. A more flexible and advanced option would be to create a formal grammar and write a parser to do this processing. This could potentially work but there are many edge cases to be considered and ambiguous or inconsistent situations would still be challenging to overcome. Currently I am considering a more probabilistic approach using machine learning. A natural language processing technique for this task I am considering is Conditional Random Fields. There are numerous libraries for this kind of task with training sets such as \href{https://taku910.github.io/crfpp/}{CRF++} that I could use but I might benefit from implementing a specific version for recipe parsing. \href{https://open.blogs.nytimes.com/2015/04/09/extracting-structured-data-from-recipes-using-conditional-random-fields/}{NYT did this for their cooking database} but it doesn't look maintained. For the sake of learning I would usually like to do what I can on my own if possible, though.

My plans for next week include:
\begin{itemize}
\item writing more parsers to expand my dataset
\item researching NLP techniques for segmenting and labeling my ingredient list
\item researching more background research related to creative computation for culinary application
\end{itemize}

My advisor hoped that I would have the recipe-ingredient occurrence graph built by our next meeting on February 6th but this might not be possible if I have to use statistical modeling to wrangle and clean my data.
\end{document}
